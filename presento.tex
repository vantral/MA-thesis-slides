\documentclass[12pt,aspectratio=169,t]{beamer}
\usepackage[english]{babel}

% Presento style file
\usepackage{config/presento}
\usepackage[fontset = none]{ctex}

\ctexset{fontset = fandol, space = false}

% custom command and packages
% custom packages
\usepackage{textpos}
\setlength{\TPHorizModule}{1cm}
\setlength{\TPVertModule}{1cm}

\newcommand\crule[1][black]{\textcolor{#1}{\rule{2cm}{2cm}}}



\usepackage{expex}
\lingset{everygla=\sffamily, belowglpreambleskip=-6pt, aboveglftskip=-12pt, belowexskip=0pt, aboveexskip=0pt, glhangstyle=none, extraglskip=0.25em, interpartskip=0.5ex, numoffset=-0.2ex} % gloss formatting
\usepackage{leipzig}
\newleipzig{Coref}{coref}{coreferent}
\newleipzig{Npst}{npst}{non-past}
\newleipzig{Ep}{ep}{epenthesis}
\newleipzig{Inch}{inch}{inchoative}
\newleipzig{Ss}{ss}{same subject}
\newleipzig{Todp}{tod.p}{today's past tense}
\newleipzig{S}{s}{singular}
\renewleipzig{Ptcp}{ptc}{participle}
\newleipzig{Nfut}{nfut}{non-future}
\newleipzig{Dir}{dir}{directive}
\renewleipzig{Incl}{inc}{inclusive}
\renewleipzig{Excl}{exc}{exclusive}
\newleipzig{Prol}{prol}{prolative}
\newleipzig{Dloc}{dloc}{directive-locative}
\newleipzig{Add}{add}{additive}
\newleipzig{Freq}{freq}{frequentative}
\newleipzig{Dim}{dim}{diminutive}
\newleipzig{ConvAnt}{convant}{~~~~~~anterior converb}
\newleipzig{Sup}{sup}{supine}
\newleipzig{Pr}{pr}{~}
\newleipzig{And}{and}{~}
\newleipzig{Impf}{impf}{imperfective}
\newleipzig{Nsg}{nsg}{non-singular}
\newcommand{\Fnsg}{\First{}\Nsg{}}
\newleipzig{A}{a}{absolutive}
\newleipzig{E}{e}{ergative}
\newleipzig{Nonactive}{nonactive}{non-active}
\newleipzig{Clitic}{clitic}{clitic}
\newleipzig{Nw}{nw}{nonwitnessed}
\newleipzig{Wp}{wp}{witnessed past tense}
\newleipzig{V}{v}{\textsc{v} class in Ingush}
\newleipzig{J}{j}{\textsc{j} class in Ingush}
\newleipzig{I}{i}{first gender in Archi}
\newleipzig{II}{ii}{second gender in Archi}
\newleipzig{III}{iii}{third gender in Archi}
\newleipzig{IV}{iv}{fourth gender in Archi}
\newleipzig{Evid}{evid}{evidentiality}
\newleipzig{Int}{int}{intensifier particle}
\newleipzig{Atr}{atr}{attributivizer}
\newleipzig{Ill}{ill}{illative}
\newleipzig{Trans}{trans}{transitional}
\newleipzig{Compl}{compl}{completion}
\newleipzig{In}{in}{inessive}
\newleipzig{Fst}{fst}{false start}
% \newleipzig{}{}{}
% \newleipzig{}{}{}
% \newleipzig{}{}{}
% \newleipzig{}{}{}
\usepackage{multicol}
\defaultfontfeatures{Ligatures=TeX,Mapping=tex-text}
\usepackage[LGRgreek,noendash]{mathastext}
\usepackage{booktabs}
\usepackage{multirow}

\usepackage[parentracker=true,
backend=biber,natbib,
hyperref=true,
bibencoding=utf8,
maxcitenames=2,
citestyle=authoryear
]{biblatex}

\newcommand\Wider[2][10em]{%
	\makebox[\linewidth][c]{%
		\begin{minipage}{\dimexpr\textwidth+#1\relax}
			\raggedright#2
		\end{minipage}%
	}%
}

\usepackage{dirtree}

\usepackage{glossaries}
\makeglossaries

\setlength{\parskip}{\baselineskip}
\usepackage{pifont}
\renewcommand{\'}{\textquotesingle}

\addbibresource{bib.bib}



% \logo{\includegraphics[height=0.09\textwidth]{images/cumtb}} 

% Information
\title{\textbf{Typology of Possessive Constructions: Comparison of Head-Marking and Dependent-Marking Patterns}}
\author{Anton Buzanov}
\institute{MA Thesis. Linguistic Theory and Language Description}
\date{June 13}


\begin{document}

% Title page
\begin{frame}[plain]
\maketitle
\end{frame}


\begin{frame}{Possessive Constructions}
    Possessive constructions are defined as comprising two noun phrases: the possessor and the possessum. Their primary function is to convey possessive relationships, which are broadly understood.
    
    \begin{multicols}{2}
    	\ex
    	\begingl
    	\glpreamble Russian (Indo-European)//
    	\gla dom-\o{} otc-a//
    	\glb house-\Nom.\Sg{} father-\Gen.\Sg{}//
    	\glft `father\'s house'//
    	\endgl 
    	\xe
    	
    	\ex
    	\begingl
    	\glpreamble Even (Tungusic)//
    	\gla etiken  ǯu-n//
    	\glb old.man house-\Poss.\Tsg{}//
    	\glft `old man\'s house'//
    	\endgl 
    	\xe
    \end{multicols}
    
\end{frame}


\begin{frame}{Locus of Marking Typology}
	Based on the specific locus of marking within possessive noun phrases, languages can be distinctly categorized into several groups \citep{nichols_locus_2013,van2016locus,van2016grammaticalization}.
	
	\begin{enumerate}
		\item Head marking (78)
		\item Dependent marking (98)
		\item Double marking (22)
		\item No marking (32)
		\item Other (6)
	\end{enumerate}
	
	\hfill \small Numbers are from \cite{nichols_locus_2013} (WALS sample).
\end{frame}

\begin{frame}{Locus of Marking Typology: Key Principle}
	\cite{nichols_locus_2013} stick to the following principles:
	
	\begin{enumerate}
		\item Selecting one construction per language (exemplar-based survey)
		\item Excluding constructions that do not allow overtly expressed possessors
		\item Excluding consideration of personal pronouns if their behaviour is different from nouns
	\end{enumerate}
\end{frame}

\begin{frame}{Locus of Marking Typology: Potential Problems}
	Mongolian is categorized as dependent marking in terms of \cite{nichols_locus_2013}.
	
	\pex
	\glpreamble Mongolian (Mongolic) from \cite{janhunen_mongolian_2012}//
	\a
	\begingl
	\gla (*min-ii) duu-men'//
	\glb \Fsg-\Gen{} younger.brother-\Poss.\First//
	\glft `my younger brother'//
	\endgl
	\a
	\begingl
	\gla min-ii eej(-*men')//
	\glb \Fsg-\Gen{} mother-\Poss.\First//
	\glft `my mother'//
	\endgl
	\xe
\end{frame}

\begin{frame}{Locus of Marking Typology: Potential Problems}
	This approach cannot capture facts provided by different constructions within the same language.
	
	Moreover, it cannot capture the variation within one construction.
	
	\begin{multicols}{2}
		\ex
		\begingl
		\glpreamble Even//
		\gla munnukan təwtə-*(n)//
		\glb hare berry-\Poss.\Tsg//
		\glft `cranberry' (lit. `hare\'s berry')//
		\endgl
		\xe
		
		\ex
		\begingl
		\glpreamble Kildin Saami//
		\gla mun kn'iga-(n)//
		\glb I.\Gen{} book-\Poss{}1//
		\glft `my book'//
		\endgl
		\xe
		
	\end{multicols}
	
\end{frame}


\begin{frame}{My Approach}
	The aim of my thesis was to create a metalanguage for a holistic description of possessive constructions in the languages of the world.
	
	\vspace{-0.5cm}
	\hfill {\footnotesize (Thanks to Mikhail Aleksandrovich for a nice wording.)}

	
\end{frame}

\begin{frame}{My Approach}
	I did not limit myself to choosing the single construction per language.
	
	I employ a bottom-up strategy, initially categorizing markers as either C- or D-marking \citep{lander2020head}, and subsequently classifying a language as a collection of pairs of these markers.
	
	\begin{itemize}
		\item D-markers = dependent markers
		\item C-markers = anywhere-but-dependent markers (Markers may occupy different places within the whole construction)
	\end{itemize}
	
	 Russian would be characterized as a language featuring several D-markers, such as \Gen{} and possessive forms of (pro)nouns, while lacking any C-markers.
	 
\end{frame}

\begin{frame}{My Approach}
	The overall list of questions is as follows. The answer to each question is ``yes'' if any construction in the language allows it.
	
	\begin{enumerate}
		\item Is juxtaposition possible?
		\item Can C-marking be used without an overt possessor?
		\item Can C-marking be used with an unmarked possessor?
		\item Can D-marking be used without C-marking?
		\item Can D-marking be used with an unmarked possessum?
	\end{enumerate}
\end{frame}

\begin{frame}{Applying the Approach}
	Constructing a \textbf{variety sample} is essential.
	
	\pause
	The aim of a variety sample is to capture a broad spectrum of linguistic patterns with minimal effort, as outlined by \cite{miestamo2016sampling}.
	
	\pause
	
	Important terms:
	
	\vspace{-0.7em}
	\begin{itemize}
		\item A \textbf{genus} refers to a grouping of languages with a time depth not exceeding 3,500 to 4,000 years \citep{dryer1989large}.
		\item \textbf{Macroareas} are continent-size linguistic areas
		which are independent of each other, but within which languages are to some
		extent typologically similar (\citealt{miestamo2016sampling} after \citealt{dryer1989large}).
	\end{itemize}
\end{frame}


\begin{frame}{Applying the Approach}
	\cite{miestamo2016sampling} suggested the Genus-Macroarea approach for variety sampling.
	\begin{itemize}
		\item Count genera in each macroarea
		\item Take as many languages as you need for your sample keeping the overall Genus-Macroarea proportion
	\end{itemize}
	
\end{frame}	
\begin{frame}{Applying the Approach}	
	Problems:
	\begin{itemize}
		\item Genera are systematically assigned only in WALS \citep{wals}.
		\item Merging WALS data with Glottolog data results in coverage of only 28.1\% with genera.
	\end{itemize}
	
	I augmented the number of languages with known genera, so the total is \textbf{75.8}\% (6,506 languages).

\end{frame}


\begin{frame}{Sample}
	I modified the algorithm of automatic sample generation proposed by \cite{cheveleva2023} and came up with the sample of 23 languages.
	
	\vspace{-0.5em}
	\footnotesize
	\begin{itemize}
		\item \textbf{Papunesia}: Marind (Marind-Yaqay), To'abaita (Oceanic), Lundayeh (North Borneo), Kobon (Kalam-Kobon), West Coast Bajau (Sama-Bajaw), Iloko (Northern Luzon)
		\item \textbf{South America}: Chácobo (Panoan), Hup (Nadahup), Yucuna (Japura-Colombia), Kwaza (isolate)
		\item \textbf{Eurasia}: Even (Tungusic), Kildin Saami (Saami), Abaza (Northwest Caucasian), Russian (Slavic), Mongolian (Mongolic)
		\item \textbf{North America}: Poqomam (Mayan), Central Alaskan Yupik (Eskimo), Haida (Haida)
		\item \textbf{Africa}: Ewe (Gbe), Ruund (Bantu), Paku (Barito), Lamang (Biu-Mandara)
		\item \textbf{Australia}: Ngardi (Western Pama-Nyungan)
	\end{itemize}
\end{frame}

\begin{frame}{Results: Languages with Single Type of Marking}
	\Wider{
	\begin{table}[h!]
		\centering
		\footnotesize
		\begin{tabular}{@{}lccccc@{}}
			\toprule
			language & juxtaposition & \parbox{1.7cm}{C-marker without possessor} & \parbox{2.35cm}{C-marker with unmarked possessor} & \parbox{1.9cm}{C-marker \& D-marker} & \parbox{2.35cm}{D-marker with unmarked possessum} \\ \midrule
			Ruund & - & NA & NA & NA & + \\
			Russian & - & NA & NA & NA & + \\
			Kwaza & - & NA & NA & NA & + \\
			Paku & + & + & - & NA & NA \\
			WC Bajau & + & + & - & NA & NA \\
			Iloko & + & + & - & NA & NA \\
			Lamang & - & + & + & NA & NA \\
			Abaza & - & + & + & NA & NA\\
			Poqomam & + & + & + & NA & NA \\
			To'abaita & + & + & + & NA & NA \\
			Kobon & + & + & + & NA & NA \\
			\bottomrule
		\end{tabular}
	\end{table}
}
\end{frame}

\begin{frame}{Results: Languages with Single Type of Marking}
	\begin{itemize}
		\item Only three languages -- Russian, Ruund, and Kwaza -- do not use any form of C-marking in the possessive domain. These languages do not allow juxtaposition to convey possessive meaning as well.
		\item No language exclusively allows a C-marker without an overt possessor; instead, juxtaposition and/or using the unmarked possessor are always options.
		\begin{itemize}
			\item Some languages allow juxtaposition while disallowing an expressed possessor
			\item Others disallow juxtaposition but permit an overt possessor
			\item Some languages allow both.
		\end{itemize}
	\end{itemize}
\end{frame}

\begin{frame}{Results: Languages with Two Types of Marking}
	\Wider{
		\begin{table}[h!]
			\centering
			\footnotesize
			\begin{tabular}{@{}lccccc@{}}
				\toprule
				language & juxtaposition & \parbox{1.7cm}{C-marker without possessor} & \parbox{2.35cm}{C-marker with unmarked possessor} & \parbox{1.9cm}{C-marker \& D-marker} & \parbox{2.35cm}{D-marker with unmarked possessum} \\ \midrule
				Haida & - & + & + & - & + \\
				Hup & - & + & + & - & + \\
				CA Yupik & - & + & - & + & + \\
				Chacobo & - & + & - & + & + \\
				Kildin Saami & - & + & - & + & + \\
				Even  & - & + & - & + & - \\
				Ngardi & - & + & - & - & +\\
				Marind & + & + & + & + & + \\
				Mongolian & + & + & - & - & +\\
				Lundayeh & + & + & - & - & + \\
				Yucuna & + & + & - & - & + \\
				Ewe & + & - & + & - & + \\
				\bottomrule
			\end{tabular}
		\end{table}
	}
\end{frame}

\begin{frame}{Results: Languages with Two Types of Marking}
	\begin{itemize}
		\item Typically, languages disallow the use of a C-marker with an unmarked possessor, given that D-marking is present in the language, with two exceptions.
		
		\begin{enumerate}
			\item If a language does not allow a C-marker without an overt possessor, it allows a C-marker with an unmarked possessor.
			\item If a language disallows both juxtaposition and the expression of both C- and D-markers, it allows a C-marker with an unmarked possessor.
		\end{enumerate}
		
		\item Typically, languages allow D- and C-markers to function on their own at least for some lexemes or constructions.
		
		\pause
		
		\begin{itemize}
			\item Two exceptions are Even and Ewe (don't mix up the two!!).
		\end{itemize}
		
	\end{itemize}
\end{frame}

\begin{frame}{Further Work}
	\begin{itemize}
		\item Incorporate the data on indexing and registration at more general level.
		\begin{itemize}
			\item Some patterns could be described with reference to this distinction.
			\item However, it might be a part of some other phenomenon, e.~g. the ability to identify features of a participant from the context.
		\end{itemize}
		\item Investigate the juxtaposition strategy in more detail.
		\begin{itemize}
			\item Juxtaposition in my sample is treated inconsistently.
			\item Perhaps, it must be seen as an instance of head or dependent marking.
		\end{itemize}
		\item Make a larger sample
	\end{itemize}
\end{frame}

\begin{frame}{Conclusions}
	\small
	\begin{itemize}
		\setlength{\itemsep}{0pt}
		\item Primary set of criteria for systematic annotation of possessive constructions.
		\item Expanded Glottolog classifications using WALS data, annotating over 6,000 languages.
		\item Selected 23 genera for a pilot sample using the Genus-Macroarea method.
		\item Manually described the possessive system for one language from each genus.
		\item Identified spectrum of marking strategies.
		\item Analysis underscores the complexity and diversity of possessive marking strategies.
		\item Trends identified provide a foundation for future statistical analyses on larger samples.
	\end{itemize}
\end{frame}

\setbeamertemplate{frametitle continuation}

\begin{frame}[allowframebreaks,t]{References}
    \printbibliography
\end{frame}

\begin{frame}{Locus of Marking Typology: Potential Problems}
	Hungarian is categorized as head marking in terms of \cite{nichols_locus_2013}.
	
	\excnt=4
	\ex
	\begingl
	\glpreamble Hungarian adapted from \cite[263]{szabolcsi1981possessive}//
	\gla az én kar-ja-i-m//
	\glb the I arm-\Poss-\Pl-\Fsg//
	\glft `my arms'//
	\endgl
	\xe
	
	\pause
	\textit{én} is the nominative form: it is used for marking the subject.
	
	\pause
	Is this form truly unmarked? For me, the answer is no. It is \textbf{marked with nominative} since there are other cases in Hungarian.
\end{frame}

\begin{frame}{Locus of Marking Typology: Potential Problems}
	In languages with case systems, cross-linguistic variation determines whether a dedicated genitive form (as in Russian), a morphologically unmarked form (nominative -- as in Hungarian), or both (Turkish), can be used to indicate possession.
	
	However, from a syntactic perspective, each form is assigned a case value.
	
	\pause
	Thus, the notion of unmarkedness must be rethought here.
\end{frame}

\begin{frame}{My Approach: Problems with Juxtaposition}
	I acknowledge that my understanding of juxtaposition is inconsistent throughout my thesis. However, the concept itself might be problematic and requires a lot of further investigation.
	
	\textsc{possessor + possessum} is considered juxtaposition if both the possessor and the possessum are morphologically unmarked, and one of the following conditions is met:
	
	\begin{itemize}
		\item The language does not have cases.
		\item There is an option to morphologically mark the possessor.
	\end{itemize}
	
\end{frame}


\begin{frame}{Applying the Approach}
	\begin{columns}
		\column{0.7\textwidth}
		\begin{itemize}
			\item Genera are systematically assigned only in WALS \citep{wals}.
			\item Merging WALS data with Glottolog data results in coverage of only 28.1\% with genera.
			\item There is potential to increase the number of languages with known genera.
		\end{itemize}
		
		\column{0.3\textwidth}
		\pause
		
		\dirtree{%
			.1 X$_1$.
			.2 X$_2$.
			.3 X$_3$.
			.3 X$_4$.
			.3 \textcolor{red}{X$_5$}.
			.2 X$_6$.
			.3 \textcolor{red}{X$_7$}.
			.3 X$_8$.
		}
		
	\end{columns}
	
	\pause
	
	After implementing the algorithm and accounting for isolates not included in WALS, the coverage increased significantly to \textbf{75.8} percent. In practical terms, this means that I successfully annotated \textbf{6,506} languages from Glottolog with their respective genera.
	
\end{frame}

\begin{frame}{Algorithm}
	
	\begin{itemize}
		\item Modified the algorithm proposed by \cite{cheveleva2023}.
		\item Retained automatic suggestion of genera but introduced a more manual selection process.
		\item Algorithm now suggests genera for closer examination, empowering linguists in language selection.
		\item Created individual tables for each genus:
		\begin{itemize}
			\item Lengths of available grammars.
			\item Rationale for assigning languages to specific genera.
		\end{itemize}
		\item Included details on:
		\begin{itemize}
			\item Common ancestor when assigning based on related languages.
			\item Related languages for languages sharing a common ancestor.
		\end{itemize}
	\end{itemize}
\end{frame}

\begin{frame}{Ewe}
	Ewe possessive constructions can be expressed through several methods:
	
	\begin{itemize}
		\item \textbf{Possessive Linker}: This method involves a possessive linker \textit{fe}, and is used in the `alienable nominal construction' where the structure is NP$_{possessor}$ fe NP$_{possessum}$.
		\item \textbf{Juxtaposition}: This method, known as the `inalienable nominal construction', also follows the structure NP$_{possessor}$ NP$_{possessum}$ without any connective.
		
		\item \textbf{Syntactic Compounding}: In this structure, the two nominals are compounded and marked with a high tone suffix at the end: N$_{possessor}$-N$_{possessum}$ + HIGH TONE SUFFIX.
	\end{itemize}
	
\end{frame}


\begin{frame}{Kwaza}
	The possessor -- the modifier in the Kwaza possessive construction -- is a personal pronoun or a noun. These constructions require a derivational possessive morpheme \textit{-dy-}, which must be applied to the possessor and which must be followed by a classifier, usually the nominaliser \textit{-hy}, which functions as a semantically neutral classifier, as exemplified in (\ref{buttock}).
	
	\ex
	\begingl\label{buttock}
	\gla 'si-dy-hy ecui'ri //
	\glb I-POS-NOM buttock //
	\glft `my buttock' //
	\endgl
	\xe

\end{frame}

\begin{frame}{Kwaza}
	Sometimes a choice is possible between a Neutral or a more specific classifier. When the specific classifier is etymologically related to the referent it classifies, the referent may be omitted, as in (\ref{omitted}), as if the classifier were a cross-reference morpheme.
	
	\ex
	\begingl \label{omitted}
	\gla (a'xy) 'si-dy-xy //
	\glb (house) I-POS-CL:house //
	\glft `my house' //
	\endgl
	\xe
	
	This is the only language in the sample that allows for omitting the possessum. From one hand, it could be analyzed in a way that the classifier morpheme is the possessum itself; however, it seems odd for me.
\end{frame}

\begin{frame}[allowframebreaks,t]{Other Parameters}
	\small 
	There are three other parameters that worth mentioning:
	
	\begin{enumerate}
		\item Indexation properties: C-markers often vary based on the $\phi$-features of the possessor, while D-markers typically do not.
		\item Omission properties: C-markers can be used with omitted possessors, whereas D-markers cannot be used with omitted possessums.
		\item Reflexive properties: Reflexive C-markers are rarely found.
	\end{enumerate}
	
	
	C-markers typically express the person value of the possessor, often reflecting number as well, i.~e. they are indexing. Among languages in the sample, only three languages have invariable C-markers: Ewe, Ngardi, and Haida. However, in Ngardi and Haida, these markers are almost always used in contexts of third-person reference, except when the possessor is obvious from the previous context. This implies that these markers are \textit{almost} indexing.
	
	In contrast, D-markers are almost never indexing but rather registering. Among the languages in the sample, only Ruund and Kwaza have indexing D-marking strategies, both employing possessive classifiers.
	
	This difference leads to distinct patterns for C- and D-markers. Almost all C-markers can be used without an explicitly mentioned possessor, as they already index all its features. Conversely, only one D-marker can be used without an overt possessum: the one in Kwaza. Furthermore, possessums cannot be easily dropped without altering the distribution of the entire noun phrase since they are the heads. While it's possible to form a headless possessive noun phrase in contexts of ellipsis or predicative possession, I have not found mentions of omitting the possessum without changes in distribution or semantics for any language except Kwaza.
	
\end{frame}

\end{document}